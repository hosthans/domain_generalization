\section{Introduction}

The rapid advancement of artificial intelligence and machine learning has led to the development of increasingly complex models, which are often trained on large datasets. Performing well when the test data distribution matches closely the training data distribution. When those models are exposed to real-world scenarios, with inputs differing in distribution due changes in style, context, domain semanctics or acquisition conditions - so being exposed to out-of-distribution (OOD) data - their performance often degrades. This vulnerability to domain shifts poses a crticial challenge for deploying models in real-world applications such as autonomous driving or medical imaging, where OOD is rather the norm than the exception.

Domaing Generalization addresses the OOD challence by training models on multiple source domains with the goal of learning representation that generalizes well even to unseen target domains. With their target domain being the OOD data, the goal is to learn domain-invariant features that are robust to the OOD data.

While existing approaches relying on meta-learning or domain alignment have shown promise, they often require a complex, ressource intensive training pipeline. In contrast, our project explores MixStyle \citep{zhouMixStyleNeuralNetworks2023} a novel and lightweight approach, yet effective feature-space data augmentation technique, trying to mitiage the effects of OOD data. By perturbing style statisitcs in the early layers of a convolutional neural network, MixStyle simulates domain variablity and encourages the model to learn domain-invariant features.

We will evaluate MixStyle on the PACS and Office-Home datasets using ResNet-18 and ResNet-50 architectures as backbone. Additionally, our experiments also include configurations with and without layer freezing, various data splits and hyperparameter tuning. The results suggest that MixStyle improves the generalization of a model, especially under the tested limited datasets. Thereby, offering a practical and easy to implement solution to strengthen the robustness of a model.